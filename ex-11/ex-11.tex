\documentclass[a4paper, titlepage]{article}
\usepackage[round, sort, numbers]{natbib}
\usepackage[utf8]{inputenc}
\usepackage{amsfonts, amsmath, amssymb, amsthm}
\usepackage{color}
\usepackage{listings}
\usepackage{mathtools}
\usepackage{multicol}
\usepackage{paralist}
\usepackage{parskip}
\usepackage{subfig}
\usepackage{tikz}
\usepackage{titlesec}

\numberwithin{figure}{section}
\numberwithin{table}{section}

\usetikzlibrary{arrows, automata, backgrounds, petri, positioning}
\tikzstyle{place}=[circle, draw=blue!50, fill=blue!20, thick]
\tikzstyle{marking}=[circle, draw=blue!50, thick, align=center]
\tikzstyle{transition}=[rectangle, draw=black!50, fill=black!20, thick]

% define new commands for sets and tuple
\newcommand{\setof}[1]{\ensuremath{\left \{ #1 \right \}}}
\newcommand{\tuple}[1]{\ensuremath{\left \langle #1 \right \rangle }}
\newcommand{\card}[1]{\ensuremath{\left \vert #1 \right \vert }}

\definecolor{lstbg}{rgb}{1,1,0.9}
\lstset{basicstyle=\ttfamily, numberstyle=\tiny, breaklines=true, backgroundcolor=\color{lstbg}, frame=single}
\lstset{language=C}

\makeatletter
\newcommand\objective[1]{\def\@objective{#1}}
\newcommand{\makecustomtitle}{%
	\begin{center}
		\huge\@title \\
		[1ex]\small Dimitri Racordon \\ \@date
	\end{center}
	\@objective
}
\makeatother

\begin{document}

  \title{Outils formels de Modélisation \\ 11\textsuperscript{ème} séance d'exercices}
  \author{Dimitri Racordon}
  \date{08.12.17}
	\objective{
    Dans cette séance d'exercices, nous allons étudier la logique du premier ordre.
    Nous tâcherons notamment de traduire des phrases d'un domaine externe à la logique
    en formule du premier ordre, et inversément.
  }

	\makecustomtitle

  \section{logic.translate.com?fr-fol ($\bigstar$)}
    Traduisez les phrases suivantes en formules de logique du premier ordre.
    Créez autant de prédicats et/ou de formules que nécessaire.
    \begin{enumerate}
      \item Il y a un assistant qui n'est pas une femme.
      \item Tous les étudiants sont soit des hommes, soit des femmes.
      \item Tous les étudiants qui étudient les méthodes formelles sont intelligents.
      \item Aucun étudiant n'est meilleur que tous les autres étudiants.
    \end{enumerate}

  \section{logic.translate.com?fol-fr ($\bigstar\bigstar$)}
    Traduisez les formules suivantes en phrases.
    \begin{enumerate}
      \item $\forall a, (Homme(a) \implies Barbe(a)) \land (Femme(a) \implies \lnot Barbe(a))$
      \item $\exists a, \exists b, \exists c, Soeur(a,b) \land Soeur(b,c) \land Soeur(c,a)$
      \item $\forall x, \forall y, BelleSoeur(x,y) \implies \exists z, Femme(x) \land Epouse(y,z) \land Soeur(x,z)$
      \item $\forall x, \forall y, Enfant(x) \land Pokemon(y) \implies Aime(x,y)$
    \end{enumerate}

  \pagebreak

  \section{Un peu de Swift ($\bigstar\bigstar\bigstar$)}
    La logique du premier ordre permet exprime des formules sur des ensembles de tailles arbitraires,
    éventuellement inifinies,
    par l'utilisation du quantificateur existentiel ($\exists$) et universel ($\forall$).
    En Swift, on peut traduire cela par l'utilisation de prédicats sur des séquences:

    \begin{lstlisting}
    func isWomanName(name: String) -> Bool {
      switch name {
      case "Aline"  : return true
      case "Cynthia": return true
      default       : return false
      }
    }

    let people = ["Aline", "Bernard", "Cynthia"]
    let thereAreWoman     = people.contains(isWoman)
    let thereAreOnlyWoman = people.filter(isWoman).count == people.count
    \end{lstlisting}

    En partant de ce principe, écrivez en Swift les formules de logique des exercices 1 et 2.

\end{document}
